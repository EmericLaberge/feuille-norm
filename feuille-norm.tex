\documentclass[french]{article}
% letter paper 
\usepackage[landscape]{geometry}
\usepackage{url}
\usepackage{mathrsfs}
\usepackage{multicol}
\usepackage{amsmath} \usepackage{esint}
\usepackage{amsfonts}
\usepackage{pgfplots}
\usepackage{svg}
\pgfplotsset{compat=newest}
\usepackage{tikz}
\usetikzlibrary{decorations.pathmorphing}
\usepackage{amsmath,amssymb}
\usetikzlibrary{arrows,shapes,positioning,graphs}
% use utf8 encoding
\usepackage[T1]{fontenc}
\usepackage[utf8]{inputenc}
\usepackage{graphicx} % Required for inserting images
\usepackage{amsmath,amssymb,amsthm}
\usepackage{amsfonts}
\usepackage{babel}
\usepackage[utf8]{inputenc}
\usepackage{xcolor}
\usepackage{tikz}
% \usepackage{algorithm}
\usepackage[ruled,vlined,linesnumbered]{algorithm2e}
\renewcommand{\qedsymbol}{$\blacksquare$}

\usepackage{colortbl}
\usepackage{multicol}
\usepackage{xcolor}
\usepackage{mathtools}
\usepackage{amsmath,amssymb}
\usepackage{enumitem}
\usepackage{bbding}
\usepackage{pifont}
\usepackage{wasysym}
\usepackage{amssymb}
\usepackage{qtree}

\makeatletter
\newcommand*\bigcdot{\mathpalette\bigcdot@{.5}}
\newcommand*\bigcdot@[2]{\mathbin{\vcenter{\hbox{\scalebox{#2}{$\m@th#1\bullet$}}}}}
\newcommand{\titledbox}[2]{%
    \begin{tikzpicture}
        \node [draw=black, fill=white, very thick,
               rectangle, rounded corners, inner sep=10pt, inner ysep=10pt] (box) {
            \begin{minipage}{0.3\textwidth}
                {\small #2}
            \end{minipage}
        };
        \node[fill=black, text=white, font=\bfseries, right=10pt] at (box.north west) {#1};
    \end{tikzpicture}%
}

\newcommand{\tb}[2]{\titledbox{#1}{#2}}
\makeatother

\title{Feuille de note normalisation}

\advance\topmargin-.8in
\advance\textheight3in
\advance\textwidth2in \advance\oddsidemargin-1in \advance\evensidemargin-1in
\parindent0pt
\parskip2pt
\newcommand{\hr}{\centerline{\rule{3.5in}{1pt}}}

\begin{document}

\begin{center}{\huge{\textbf{Feuille de note normalisation}}}\\
\end{center} \begin{multicols*}{3}

	\tb{Réflexivité}{
		Soit $X$ et $Y$ deux ensembles d'attributs, alors
		$$ \text{Si } Y \subseteq X \text{ alors } X \rightarrow Y $$
		Exemples :
		\begin{align*}
			A   & \rightarrow A & A,B & \rightarrow A,B \\
			A,B & \rightarrow A & A,B & \rightarrow B
		\end{align*}
	}
	\tb{Augmentation}{
		Soit $X,Y,Z$ trois ensembles d'attributs, alors
		$$ X \rightarrow Y \Rightarrow XZ \rightarrow YZ $$
		Exemples :
		\begin{align*}
			A   & \rightarrow B \Rightarrow A,C \rightarrow B,C   \\
			B,C & \rightarrow A \Rightarrow B,C,E \rightarrow A,E
		\end{align*}
	}
	\tb{Transitivité}{
		Soit $X,Y,Z$ trois ensembles d'attributs, alors
		$$ X \rightarrow Y \text{ et } Y \rightarrow Z \Rightarrow X \rightarrow Z $$
		Exemples :
		\begin{align*}
			A   & \rightarrow B \text{ et } B \rightarrow C  \boldsymbol{\Rightarrow} A \rightarrow C      \\
			B,C & \rightarrow C,E \text{ et } C,E \rightarrow D \boldsymbol{\Rightarrow} B,C \rightarrow D
		\end{align*}
	}
	\tb{Décomposition}{
		Soit $X,Y,Z$ trois ensembles d'attributs, alors
		$$ X \rightarrow YZ \Rightarrow X \rightarrow Y \text{ et } X \rightarrow Z $$
		Exemple:
		\begin{align*}
			A,B & \rightarrow C,D \Rightarrow A,B \rightarrow C \text{ et } A,B \rightarrow D \\
		\end{align*}}
	\tb{Composition}{
		Soit $X,Y,A,B$ quatre ensembles d'attributs, alors
		$$ X \rightarrow Y \text{ et } A \rightarrow B \Rightarrow XA \rightarrow YB $$
		Exemple:
		\begin{align*}
			A & \rightarrow B \text{ et } C \rightarrow D \Rightarrow AC \rightarrow BD \\
		\end{align*}}

	\vspace{1cm}

	\tb{Union}{
		Soit $X,Y,Z$ trois ensembles d'attributs, alors
		$$ X \rightarrow Y \text{ et } X \rightarrow Z \Rightarrow X \rightarrow YZ $$
	}
	\tb{Pseudo-transitivité}{
		Soit $X,Y,Z,W$ quatre ensembles d'attributs, alors
		$$ X \rightarrow Y \text{ et } YZ \rightarrow W \Rightarrow XZ \rightarrow W $$
		Exemple:
		\begin{align*}
			A & \rightarrow B \text{ et } BC \rightarrow D \Rightarrow AC \rightarrow D \\
		\end{align*}
	}
	\tb{Détermination de soi (Self-determination)}{
		$$ I \rightarrow I \quad \forall I $$
	}
	\tb{Extensivité}{
		Soit $X,Y,Z$ trois ensembles d'attributs, alors
		$$ X \rightarrow Y \Rightarrow X \rightarrow XY $$
	}
	\tb{Fermeture de \mathscr{F}}{
		Soit $\mathscr{F}$ un ensemble de dépendances fonctionnelles, alors
		$$ \mathscr{F}^+ = \{ f \text{ tel que } \mathscr{F} \models f \} $$
		Autrement dit, $\mathscr{F}^+$ est l'ensemble de toutes les dépendances fonctionnelles
		qui sont logiquement impliquées par $\mathscr{F}$ en utilisant les axiomes d'Armstrong.
	}
	\tb{Fermeture de X}{
		Soit $X$ un ensemble d'attributs et $\mathscr{F}$ un ensemble de dépendances fonctionnelles, alors
		$$ X^+ = \{ A \text{ tel que } X \rightarrow A \text{ est dérivable de } \mathscr{F} \} $$
	}
	\tb{Clés et superclés}{
		Soit $X$ un ensemble d'attributs et $\mathscr{F}$ un ensemble de dépendances
		fonctionnelles\\

		\textbf{Superclé:}\\
		$X$ est une superclé ssi
		$$ X^+ = R $$

    \vspace{-8pt}

		\textbf{Clé:}\\
		$X$ est une clé ssi
    \vspace{-6pt}
		$$ X^+ = R \text{ et } \nexists Y \subset X \text{ tel que } Y^+ = R $$
	}
	\tb{Dépendance fonctionnelle élémentaire}{
		Soit $X$ un ensemble d'attributs et $A$ un attribut, alors $X \rightarrow A$ est élémentaire ssi
		\begin{enumerate}
			\item $A \notin X$,
			\item $\nexists X' \subset X \text{ tel que } X' \rightarrow Y$,
			      c-à-d, que on ne peut pas trouver un sous-ensemble de $X$ qui détermine $Y$.
		\end{enumerate}
    Note: La partie de droite ne peut pas être un groupe d'attributs.
	}
	\tb{Couverure minimale}{
		Soit $\mathscr{F}$ un ensemble de dépendances fonctionnelles, alors
		$\mathscr{F}$ est minimale ssi on ne peut pas enlever une dépendance
		fonctionnelle sans perdre de l'information.\\
	}
	\tb{Forme normale de Boyce-Codd (BCNF)}{
		Une relation $R$ est en BCNF ssi pour chaque DF non triviale $X \rightarrow
			A$, on a que:
		$$ X \text{ est une superclé de } R $$
	}
	\tb{1FN,2FN,3FN}{

		\textbf{1FN:}\\
		Chaque attribut est atomique, c-à-d, chaque attribut ne contient qu'une seule valeur.\\
		\hline\\
		\vspace{8pt}
		\textbf{2FN:}\\
		Pour être en 2FN, il faut que:
		\begin{enumerate}
			\item Il faut être en 1FN
			\item Chaque attribut non-clé ne dépend d'une partie de la clé.
		\end{enumerate}
		\hline\\
		\vspace{8pt}
		\textbf{3FN:}\\
		Pour être en 3FN, il faut que:
		\begin{enumerate}
			\item Il faut être en 2FN
			\item Chaque attribut non-clé ne dépend pas d'un attribut
			      non-clé.
		\end{enumerate}
	}



\end{multicols*}
\end{document}
% From Wikipedia, the free encyclopedia
%
% Armstrong's axioms are a set of axioms (or, more precisely, inference rules) used to infer all the functional dependencies on a relational database. They were developed by William W. Armstrong in his 1974 paper.[1] The axioms are sound in generating only functional dependencies in the closure of a set of functional dependencies (denoted as \mathscr{F} + {\displaystyle \mathscr{F}^{+}}) when applied to that set (denoted as \mathscr{F} {\displaystyle \mathscr{F}}). They are also complete in that repeated application of these rules will generate all functional dependencies in the closure \mathscr{F} + {\displaystyle \mathscr{F}^{+}}.
%
% More formally, let ⟨ R ( U ) , \mathscr{F} ⟩ {\displaystyle \langle R(U),\mathscr{F}\rangle } denote a relational scheme over the set of attributes U {\displaystyle U} with a set of functional dependencies \mathscr{F} {\displaystyle \mathscr{F}}. We say that a functional dependency f {\displaystyle f} is logically implied by \mathscr{F} {\displaystyle \mathscr{F}}, and denote it with \mathscr{F} ⊨ f {\displaystyle \mathscr{F}\models f} if and only if for every instance r {\displaystyle r} of R {\displaystyle R} that satisfies the functional dependencies in \mathscr{F} {\displaystyle \mathscr{F}}, r {\displaystyle r} also satisfies f {\displaystyle f}. We denote by \mathscr{F} + {\displaystyle \mathscr{F}^{+}} the set of all functional dependencies that are logically implied by \mathscr{F} {\displaystyle \mathscr{F}}.
%
% Furthermore, with respect to a set of inference rules A {\displaystyle A}, we say that a functional dependency f {\displaystyle f} is derivable from the functional dependencies in \mathscr{F} {\displaystyle \mathscr{F}} by the set of inference rules A {\displaystyle A}, and we denote it by \mathscr{F} ⊢ A f {\displaystyle \mathscr{F}\vdash _{A}f} if and only if f {\displaystyle f} is obtainable by means of repeatedly applying the inference rules in A {\displaystyle A} to functional dependencies in \mathscr{F} {\displaystyle \mathscr{F}}. We denote by \mathscr{F} A ∗ {\displaystyle F_{A}^{*}} the set of all functional dependencies that are derivable from \mathscr{F} {\displaystyle \mathscr{F}} by inference rules in A {\displaystyle A}.
%
% Then, a set of inference rules A {\displaystyle A} is sound if and only if the following holds:
%
% \mathscr{F} A ∗ ⊆ \mathscr{F} + {\displaystyle F_{A}^{*}\subseteq \mathscr{F}^{+}}
%
% that is to say, we cannot derive by means of A {\displaystyle A} functional dependencies that are not logically implied by \mathscr{F} {\displaystyle \mathscr{F}}. The set of inference rules A {\displaystyle A} is said to be complete if the following holds:
%
% \mathscr{F} + ⊆ \mathscr{F} A ∗ {\displaystyle \mathscr{F}^{+}\subseteq F_{A}^{*}}
%
% more simply put, we are able to derive by A {\displaystyle A} all the functional dependencies that are logically implied by \mathscr{F} {\displaystyle \mathscr{F}}.
% Axioms (primary rules)
%
% Let R ( U ) {\displaystyle R(U)} be a relation scheme over the set of attributes U {\displaystyle U}. Henceforth we will denote by letters X {\displaystyle X}, Y {\displaystyle Y}, Z {\displaystyle Z} any subset of U {\displaystyle U} and, for short, the union of two sets of attributes X {\displaystyle X} and Y {\displaystyle Y} by X Y {\displaystyle XY} instead of the usual X ∪ Y {\displaystyle X\cup Y}; this notation is rather standard in database theory when dealing with sets of attributes.
% Axiom of reflexivity
%
% If X {\displaystyle X} is a set of attributes and Y {\displaystyle Y} is a subset of X {\displaystyle X}, then X {\displaystyle X} holds Y {\displaystyle Y}. Hereby, X {\displaystyle X} holds Y {\displaystyle Y} [ X → Y {\displaystyle X\to Y}] means that X {\displaystyle X} functionally determines Y {\displaystyle Y}.
%
% If Y ⊆ X {\displaystyle Y\subseteq X} then X → Y {\displaystyle X\to Y}.
%
% Axiom of augmentation
%
% If X {\displaystyle X} holds Y {\displaystyle Y} and Z {\displaystyle Z} is a set of attributes, then X Z {\displaystyle XZ} holds Y Z {\displaystyle YZ}. It means that attribute in dependencies does not change the basic dependencies.
%
% If X → Y {\displaystyle X\to Y}, then X Z → Y Z {\displaystyle XZ\to YZ} for any Z {\displaystyle Z}.
%
% Axiom of transitivity
%
% If X {\displaystyle X} holds Y {\displaystyle Y} and Y {\displaystyle Y} holds Z {\displaystyle Z}, then X {\displaystyle X} holds Z {\displaystyle Z}.
%
% If X → Y {\displaystyle X\to Y} and Y → Z {\displaystyle Y\to Z}, then X → Z {\displaystyle X\to Z}.
%
% Additional rules (Secondary Rules)
%
% These rules can be derived from the above axioms.
% Decomposition
%
% If X → Y Z {\displaystyle X\to YZ} then X → Y {\displaystyle X\to Y} and X → Z {\displaystyle X\to Z}.
% Proof
% 1. X → Y Z {\displaystyle X\to YZ} 	(Given)
% 2. Y Z → Y {\displaystyle YZ\to Y} 	(Reflexivity)
% 3. X → Y {\displaystyle X\to Y} 	(Transitivity of 1 & 2)
% Composition
%
% If X → Y {\displaystyle X\to Y} and A → B {\displaystyle A\to B} then X A → Y B {\displaystyle XA\to YB}.
% Proof
% 1. X → Y {\displaystyle X\to Y} 	(Given)
% 2. A → B {\displaystyle A\to B} 	(Given)
% 3. X A → Y A {\displaystyle XA\to YA} 	(Augmentation of 1 & A)
% 4. Y A → Y B {\displaystyle YA\to YB} 	(Augmentation of 2 & Y)
% 5. X A → Y B {\displaystyle XA\to YB} 	(Transitivity of 3 and 4)
% Union
%
% If X → Y {\displaystyle X\to Y} and X → Z {\displaystyle X\to Z} then X → Y Z {\displaystyle X\to YZ}.
% Proof
% 1. X → Y {\displaystyle X\to Y} 	(Given)
% 2. X → Z {\displaystyle X\to Z} 	(Given)
% 3. X → X Z {\displaystyle X\to XZ} 	(Augmentation of 2 & X)
% 4. X Z → Y Z {\displaystyle XZ\to YZ} 	(Augmentation of 1 & Z)
% 5. X → Y Z {\displaystyle X\to YZ} 	(Transitivity of 3 and 4)
% Pseudo transitivity
%
% If X → Y {\displaystyle X\to Y} and Y Z → W {\displaystyle YZ\to W} then X Z → W {\displaystyle XZ\to W}.
% Proof
% 1. X → Y {\displaystyle X\to Y} 	(Given)
% 2. Y Z → W {\displaystyle YZ\to W} 	(Given)
% 3. X Z → Y Z {\displaystyle XZ\to YZ} 	(Augmentation of 1 & Z)
% 4. X Z → W {\displaystyle XZ\to W} 	(Transitivity of 3 and 2)
% Self determination
%
% I → I {\displaystyle I\to I} for any I {\displaystyle I}. This follows directly from the axiom of reflexivity.
% Extensivity
%
% The following property is a special case of augmentation when Z = X {\displaystyle Z=X}.
%
% If X → Y {\displaystyle X\to Y}, then X → X Y {\displaystyle X\to XY}.
%
% Extensivity can replace augmentation as axiom in the sense that augmentation can be proved from extensivity together with the other axioms.
% Proof
% 1. X Z → X {\displaystyle XZ\to X} 	(Reflexivity)
% 2. X → Y {\displaystyle X\to Y} 	(Given)
% 3. X Z → Y {\displaystyle XZ\to Y} 	(Transitivity of 1 & 2)
% 4. X Z → X Y Z {\displaystyle XZ\to XYZ} 	(Extensivity of 3)
% 5. X Y Z → Y Z {\displaystyle XYZ\to YZ} 	(Reflexivity)
% 6. X Z → Y Z {\displaystyle XZ\to YZ} 	(Transitivity of 4 & 5)
% Armstrong relation
%
% Given a set of functional dependencies \mathscr{F} {\displaystyle \mathscr{F}}, an Armstrong relation is a relation which satisfies all the functional dependencies in the closure \mathscr{F} + {\displaystyle \mathscr{F}^{+}} and only those dependencies. Unfortunately, the minimum-size Armstrong relation for a given set of dependencies can have a size which is an exponential function of the number of attributes in the dependencies considered.[2]
% References
%
% William Ward Armstrong: Dependency Structures of Data Base Relationships, page 580-583. IFIP Congress, 1974.
%
% Beeri, C.; Dowd, M.; Fagin, R.; Statman, R. (1984). "On the Structure of Armstrong Relations for Functional Dependencies" (PDF). Journal of the ACM. 31: 30–46. CiteSeerX 10.1.1.68.9320. doi:10.1145/2422.322414. Archived from the original (PDF) on 2018-07-23.
%
% External links
%
% UMBC CMSC 461 Spring '99
% CS345 Lecture Notes from Stanford University
%
% vte
%
% Database management systems
%
% vte
%
% Database normalization
% Categories:
%
% Database normalizationData modelingDatabase management systems
%
% This page was last edited on 5 December 2024, at 20:33 (UTC).
% Text is available under the Creative Commons Attribution-ShareAlike 4.0 License; additional terms may apply. By using this site, you agree to the Terms of Use and Privacy Policy. Wikipedia® is a registered trademark of the Wikimedia Foundation, Inc., a non-profit organization.
