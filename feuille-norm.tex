\documentclass[french]{article}
% letter paper 
\usepackage[landscape]{geometry}
\usepackage{url}
\usepackage{mathrsfs}
\usepackage{multicol}
\usepackage{amsmath} \usepackage{esint}
\usepackage{amsfonts}
\usepackage{pgfplots}
\usepackage{svg}
\pgfplotsset{compat=newest}
\usepackage{tikz}
\usetikzlibrary{decorations.pathmorphing}
\usepackage{amsmath,amssymb}
\usetikzlibrary{arrows,shapes,positioning,graphs}
% use utf8 encoding
\usepackage[T1]{fontenc}
\usepackage[utf8]{inputenc}
\usepackage{graphicx} % Required for inserting images
\usepackage{amsmath,amssymb,amsthm}
\usepackage{amsfonts}
\usepackage{babel}
\usepackage[utf8]{inputenc}
\usepackage{xcolor}
\usepackage{tikz}
% \usepackage{algorithm}
\usepackage[ruled,vlined]{algorithm2e}
\renewcommand{\qedsymbol}{$\blacksquare$}

\usepackage{colortbl}
\usepackage{multicol}
\usepackage{xcolor}
\usepackage{mathtools}
\usepackage{amsmath,amssymb}
\usepackage{enumitem}
\usepackage{bbding}
\usepackage{pifont}
\usepackage{wasysym}
\usepackage{amssymb}
\usepackage{qtree}

\makeatletter
\newcommand*\bigcdot{\mathpalette\bigcdot@{.5}}
\newcommand*\bigcdot@[2]{\mathbin{\vcenter{\hbox{\scalebox{#2}{$\m@th#1\bullet$}}}}}
\newcommand{\titledbox}[2]{%
    \begin{tikzpicture}
        \node [draw=black, fill=white, very thick,
               rectangle, rounded corners, inner sep=10pt, inner ysep=10pt] (box) {
            \begin{minipage}{0.3\textwidth}
                {\small #2}
            \end{minipage}
        };
        \node[fill=black, text=white, font=\bfseries, right=10pt] at (box.north west) {#1};
    \end{tikzpicture}%
}

\newcommand{\tb}[2]{\titledbox{#1}{#2}}
\makeatother

\title{Feuille de note normalisation}

\advance\topmargin-.8in
\advance\textheight3in
\advance\textwidth2in \advance\oddsidemargin-1in \advance\evensidemargin-1in
\parindent0pt
\parskip2pt
\newcommand{\hr}{\centerline{\rule{3.5in}{1pt}}}

\begin{document}

\begin{center}{\huge{\textbf{Feuille de note normalisation}}}\\
\end{center} \begin{multicols*}{3}

	\tb{Réflexivité}{
		Soit $X$ et $Y$ deux ensembles d'attributs, alors
		$$ \text{Si } Y \subseteq X \text{ alors } X \rightarrow Y $$
		Exemples :
		\begin{align*}
			A   & \rightarrow A & A,B & \rightarrow A,B \\
			A,B & \rightarrow A & A,B & \rightarrow B
		\end{align*}
	}
	\tb{Augmentation}{
		Soit $X,Y,Z$ trois ensembles d'attributs, alors
		$$ X \rightarrow Y \Rightarrow XZ \rightarrow YZ $$
		Exemples :
		\begin{align*}
			A   & \rightarrow B \Rightarrow A,C \rightarrow B,C   \\
			B,C & \rightarrow A \Rightarrow B,C,E \rightarrow A,E
		\end{align*}
	}
	\tb{Transitivité}{
		Soit $X,Y,Z$ trois ensembles d'attributs, alors
		$$ X \rightarrow Y \text{ et } Y \rightarrow Z \Rightarrow X \rightarrow Z $$
		Exemples :
		\begin{align*}
			A   & \rightarrow B \text{ et } B \rightarrow C  \boldsymbol{\Rightarrow} A \rightarrow C      \\
			B,C & \rightarrow C,E \text{ et } C,E \rightarrow D \boldsymbol{\Rightarrow} B,C \rightarrow D
		\end{align*}
	}
	\tb{Décomposition}{
		Soit $X,Y,Z$ trois ensembles d'attributs, alors
		$$ X \rightarrow YZ \Rightarrow X \rightarrow Y \text{ et } X \rightarrow Z $$
		Exemple:
		\begin{align*}
			A,B & \rightarrow C,D \Rightarrow A,B \rightarrow C \text{ et } A,B \rightarrow D \\
		\end{align*}}
	\tb{Composition}{
		Soit $X,Y,A,B$ quatre ensembles d'attributs, alors
		$$ X \rightarrow Y \text{ et } A \rightarrow B \Rightarrow XA \rightarrow YB $$
		Exemple:
		\begin{align*}
			A & \rightarrow B \text{ et } C \rightarrow D \Rightarrow AC \rightarrow BD \\
		\end{align*}}

	\vspace{1cm}

	\tb{Union}{
		Soit $X,Y,Z$ trois ensembles d'attributs, alors
		$$ X \rightarrow Y \text{ et } X \rightarrow Z \Rightarrow X \rightarrow YZ $$
	}
	\tb{Pseudo-transitivité}{
		Soit $X,Y,Z,W$ quatre ensembles d'attributs, alors
		$$ X \rightarrow Y \text{ et } YZ \rightarrow W \Rightarrow XZ \rightarrow W $$
		Exemple:
		\begin{align*}
			A & \rightarrow B \text{ et } BC \rightarrow D \Rightarrow AC \rightarrow D \\
		\end{align*}
	}
	\tb{Détermination de soi (Self-determination)}{
		$$ I \rightarrow I \quad \forall I $$
	}
	\tb{Extensivité}{
		Soit $X,Y,Z$ trois ensembles d'attributs, alors
		$$ X \rightarrow Y \Rightarrow X \rightarrow XY $$
	}
	\tb{Fermeture de \mathscr{F}}{
		Soit $\mathscr{F}$ un ensemble de dépendances fonctionnelles, alors
		$$ \mathscr{F}^+ = \{ f \text{ tel que } \mathscr{F} \models f \} $$
		Autrement dit, $\mathscr{F}^+$ est l'ensemble de toutes les dépendances fonctionnelles
		qui sont logiquement impliquées par $\mathscr{F}$ en utilisant les axiomes d'Armstrong.
	}
	\tb{Fermeture de X}{
		Soit $X$ un ensemble d'attributs et $\mathscr{F}$ un ensemble de dépendances fonctionnelles, alors
		$$ X^+ = \{ A \text{ tel que } X \rightarrow A \text{ est dérivable de } \mathscr{F} \} $$
	}
	\tb{Clés et superclés}{
		Soit $X$ un ensemble d'attributs et $\mathscr{F}$ un ensemble de dépendances
		fonctionnelles\\

		\textbf{Superclé:}\\
		$X$ est une superclé ssi
		$$ X^+ = R $$

		\vspace{-8pt}

		\textbf{Clé:}\\
		$X$ est une clé ssi
		\vspace{-6pt}
		$$ X^+ = R \text{ et } \nexists Y \subset X \text{ tel que } Y^+ = R $$
	}
	\tb{DF triviale}{
		Une dépendance fonctionnelle est triviale si elle est de la formes suivante:
		$$ X \rightarrow Y \text{ où } Y \subseteq X $$
	}


	\tb{Dépendance fonctionnelle élémentaire}{
		Soit $X$ un ensemble d'attributs et $A$ un attribut, alors $X \rightarrow A$ est élémentaire ssi
		\begin{enumerate}
			\item $A \notin X$,
			\item $\nexists X' \subset X \text{ tel que } X' \rightarrow Y$,
			      c-à-d, que on ne peut pas trouver un sous-ensemble de $X$ qui détermine $Y$.
		\end{enumerate}
		Note: La partie de droite ne peut pas être un groupe d'attributs.
	}
	\tb{Couverure minimale}{
		Soit $\mathscr{F}$ un ensemble de dépendances fonctionnelles, alors
		$\mathscr{F}$ est minimale ssi on ne peut pas enlever une dépendance
		fonctionnelle sans perdre de l'information.\\
	}
	\tb{Calculer la couverture minimale}{
		\begin{enumerate}
			\item Décomposer chaque DF de type $X \rightarrow Y,Z$ en $X \rightarrow Y$ et $X \rightarrow Z$.
			\item Enlever les DF non-élémentaires.
			\item Enlever les DF redondantes. (Les DF qui peuvent être déduites des autres DF)
		\end{enumerate}
	}
	\tb{Forme normale de Boyce-Codd (BCNF)}{
		\textbf{Définition 1 :}\\
		Un schéma relationnel $R(U)$ avec ensemble \mathscr{F} de DF est en forme
		BCNF ssi pour chaque dépendance élémentaire $X \rightarrow A$ dans
		$\mathscr{F}^+$, on a que
		$$ X \text{ est une clé (candidate)} $$
		\textbf{Définition 2 :}\\
		Un schéma relationnel $R(U)$ avec ensemble $\mathscr{F}$ de DF est en forme
		BCNF ssi pour chaque DF non triviale $X \rightarrow A$ de $\mathscr{F}$,
		on a que:
		$$ X \text{ est une superclé de } R $$
	}
	\tb{1FN,2FN,3FN}{

		\textbf{1FN:}\\
		Chaque attribut est atomique, c-à-d, chaque attribut ne contient qu'une seule valeur.\\
		\hline\\
		\vspace{8pt}
		\textbf{2FN:}\\
		Pour être en 2FN, il faut que:
		\begin{enumerate}
			\item Il faut être en 1FN
			\item Chaque attribut non-clé ne dépend d'une partie de la clé.
		\end{enumerate}
		\hline\\
		\vspace{8pt}
		\textbf{3FN:}\\
		Pour être en 3FN, il faut que:
		\begin{enumerate}
			\item Il faut être en 2FN
			\item Chaque attribut non-clé ne dépend pas d'un attribut
			      non-clé.
		\end{enumerate}
	}
	\tb{Algo-BCNF}{
	\textbf{Entrée:} \\ Un schéma $R(U, \mathscr{F})$ où $U$ est l'ensemble des attributs et $\mathscr{F}$ est l'ensemble des dépendances fonctionnelles.\\
	\textbf{Sortie:}\\ Une décomposition
	{\scriptsize
	$D = \{R_1(U_1, \mathscr{F}_1), \ldots, R_n(U_n, \mathscr{F}_n)\}$
	}
	sans perte d'information avec $R_i(U_i, \mathscr{F}_i)$ en BCNF pour tout $i$ (mais perte possible de dépendances).\\


	\begin{enumerate}
		\item Initialiser $D \gets \{ R(U, \mathscr{F}) \}$
		\item Tant qu'il existe $R'(U', \mathscr{F}') \in D$ qui n'est pas en BCNF
		      \begin{enumerate}
			      \item Trouver une DF non triviale $X \!\! \rightarrow \!\! Y \in \mathscr{F}'$ tq $X \neq X^+$ et $X^+ \neq U'$
			      \item Poser $U'_1 = X^+$ et $U'_2 = X \cup (U' - X^+)$, $\mathscr{F}'_1$ les dépendances sur $U'_1$ et $\mathscr{F}'_2$ celles sur $U'_2$
			      \item Poser $U'_1 = X^+$, $U'_2 = X \cup (U' - X^+)$, $\mathscr{F}'_1$ les dépendances sur $U'_1$ et $\mathscr{F}'_2$ celles sur $U'_2$
			      \item Remplacer $R'(U')$ par ses projections sur $U'_1$ et $U'_2$:
			            $R'_1(U'_1, \mathscr{F}'_1)$ et $R'_2(U'_2, \mathscr{F}'_2)$
			            ($R' = R'_1 \bowtie R'_2$ par Heath avec $X$ déterminant, $X^+ - X$ déterminé et $U' - X^+$ résidu)
		      \end{enumerate}
	\end{enumerate}
	Le calcul de $\mathscr{F}'_1$ et $\mathscr{F}'_2$ à partir de $\mathscr{F}'$ doit se faire avec l'algorithme de projection des dépendances
	}
	\tb{Projection des dépendances fonctionnelles}{
		\textbf{Entrée:} Un schéma $R(U, \mathscr{F})$ et $U_1 \subset U$\\
		\textbf{Sortie:} Les dépendances $\mathscr{F}_1$ induites sur $R_1(U_1) = \pi_{U_1}(R)$\\
		\begin{enumerate}
			\item $\mathscr{F}_1 \gets \emptyset$ (initialisation)
			\item Pour chaque $X \subseteq U_1$
			      \begin{enumerate}
				      \item Calculer $X^+$ (par rapport à $\mathscr{F}$)
				      \item Ajouter à $\mathscr{F}_1$ toutes les $X \rightarrow \{A\}$ t.q. $A
					            \in (X^+ - X) \cap U_1$ \\ (optionnel : et t.q. on n'a pas déjà $Y \rightarrow \{A\} \in \mathscr{F}_1$ avec $Y \subset X$)
			      \end{enumerate}
			\item (optionnel) enlever de $\mathscr{F}_1$ les dépendances redondantes
			\item retourner $\mathscr{F}_1$
		\end{enumerate}
	}
	\tb{Algo de Bernstein}{
		\textbf{Entrée:} Un schéma $R$ (arbitraire) et un ensemble $\mathscr{F}$ de DF qui soit en 1NF.\\
		\textbf{Sortie:} Projections donnant des schémas $R_i$ en 3-NF pour tout $i$ sans perte d'information et préservant les dépendances.\\
		\begin{enumerate}
			\item Initialiser $P$ (ensemble de projections) à l'ensemble vide (et $i = 1$)
			\item Fixer $\mathscr{G}$ une \textbf{couverture minimale} de $\mathscr{F}$
			\item Pour chaque $X$ distinct d'une partie gauche d'une DF de $\mathscr{G}$
			      \begin{enumerate}
				      \item Faire la réunion $Y$ de tous les $\{A\}$ tels que $X
					            \rightarrow \{A\} \in \mathscr{G}$
				      \item Ajouter à $P$ la projection de $R$ sur $X,Y$ (donne
				            $R_i(\underline{X},Y) = \pi_{XY}(R)$, avec clé primaire $X$)
				      \item $i = i + 1$
			      \end{enumerate}
			\item Si aucune des projections dans $P$ ne contient une clé candidate de
			      $R$, ajouter à $P$ la projection de $R$ sur une clé candidate. ($R_i(K)
				      = \pi_K(R)$ pour $K$ clé candidate).
		\end{enumerate}
		$R$ est alors la jointure naturelle des projections obtenues. La décomposition est sans perte.

	}


\end{multicols*}
\end{document}
